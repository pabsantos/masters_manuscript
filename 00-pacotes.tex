% !TeX root = ./main.tex

% Pacotes básicos
% ----------------------------------------------------------
%\usepackage{lmodern}			% Usa a fonte Latin Modern
\usepackage[T1]{fontenc}		% Selecao de codigos de fonte.
\usepackage[utf8]{inputenc}		% Codificacao do documento (conversão automática dos acentos)
\usepackage{lastpage}			% Usado pela Ficha catalográfica
\usepackage{indentfirst}		% Indenta o primeiro parágrafo de cada seção.
\usepackage{color}		    	% Controle das cores
\usepackage{graphicx}			% Inclusão de gráficos
\usepackage{microtype} 			% para melhorias de justificação
\usepackage{ifthen}		    	% para montar condicionais
\usepackage[brazil]{babel}		% para utilizar termos em portugues
\usepackage[final]{pdfpages}    % para incluir páginas de arquivos pdf
\usepackage{lipsum}				% para geração de dummy text
\usepackage{csquotes}

%\usepackage[style=long]{glossaries}
%\usepackage{abntex2glossaries}


\usepackage{cancel} 		% permite representar o cancelamento de termos em texto ou equacoes
\usepackage{xcolor} 		% cores extendidas
\usepackage{smartdiagram}   	% gera diagramas a partir de listas
\usepackage{float} 		% Para a figura ficar na posição correta
\usepackage{textcomp} 		% supporte para fontes da Text Companion
\usepackage{longtable}		% uso de longtable
\usepackage{amsmath}		% simbolos matematicos
\usepackage{lscape}		% páginas em paisagem
\usepackage{multicol}		% mescla de colunas em tabelas
\usepackage{multirow}		% mescla de linhas em tabelas
\usepackage{newfloat} 		% criação do indice de quadros
\usepackage{caption} 		% configura legenda
	%[format=plain]
	%\renewcommand\caption[1]{%
    	%\captionsetup{font=small}	% tamanho da fonte 10pt
    	%,format=hang
 	% \caption{#1}}
	%\captionsetup{width=0.8\textwidth}
\captiondelim{-- }
\captiontitlefont{\small}
\captionnamefont{\small}

% Pacotes de citações BibLaTeX
% ----------------------------------------------------------
\usepackage[style=abnt,
	backref=true,
	repeatfields=false,
	backend=biber,
	citecounter=true,
	backrefstyle=three,
	url=true,
	language=english]{biblatex}

% Texto padrão para as referências
% ----------------------------------------------------------
\DefineBibliographyStrings{brazil}{%
	 backrefpage  = {Cited on page},		% originally "cited on page"
	 backrefpages = {Cited on pages},	% originally "cited on pages"
	 urlfrom      = {Available in},
}

% Ajusta indentação de Referencias no ToC
% ----------------------------------------------------------
\defbibheading{bay}[\bibname]{%
  \chapter*{#1}%
  \markboth{#1}{#1}%
  \addcontentsline{toc}{chapter}
  {\protect\numberline{}\bibname}
}

% Formatando o avançao dos títulos no sumário
% ----------------------------------------------------------
\makeatletter
	\pretocmd{\chapter}{\addtocontents{toc}{\protect\addvspace{-12\p@}}}{}{}
	\pretocmd{\section}{\addtocontents{toc}{\protect\addvspace{-3\p@}}}{}{}
\makeatother

% Para retirar os símbolos <> da URL
% ----------------------------------------------------------
\DeclareFieldFormat{illustrated}{\addspace #1\isdot}%
%\DeclareFieldFormat{url}{\bibstring{urlform}\addcolon\addspace<\url{#1}>}%
%\DeclareFieldFormat{url}{\bibstring{urlfrom}\addcolon\addspace<\url{#1}>}%
\DeclareFieldFormat{url}{\bibstring{urlfrom}\addcolon \space\addspace{#1}}
% remove <> em urls de acordo com abnt-6023:2018

% Ajustar o espaço para a formatação da data
% ----------------------------------------------------------
\DeclareFieldFormat{urldate}{\bibstring{urlseen}\addcolon\addspace #1}%
\DeclareFieldFormat*{note}{\addspace #1}%

% Para ajustar o tamanho da fonte do número da primeira página do capítulo
% comando utilizado na parte textual
% ----------------------------------------------------------
\makepagestyle{chapfirst}% Just for the first page of a chapter
\makeoddhead{chapfirst}{}{}{\footnotesize{\thepage}}

%%criar um novo estilo de cabeçalhos e rodapés
\makepagestyle{simplestextual}
  %%cabeçalhos
  \makeevenhead{simplestextual} %%pagina par
     {}{}{\footnotesize \thepage}

  \makeoddhead{simplestextual} %%pagina ímpar ou com oneside
     {}{}{\footnotesize \thepage}
  %\makeheadrule{simplestextual}{\textwidth}{\normalrulethickness} %linha
  %% rodapé
  \makeevenfoot{simplestextual}
     {}{}{} %%pagina par

  \makeoddfoot{simplestextual} %%pagina ímpar ou com oneside
     {}{}{}

% Define a formatação dos capítulos póstextuais numerados
% ----------------------------------------------------------
\newcommand{\poschap}[1]{
	\stepcounter{chapter}
	\markboth{#1}{#1}%
	\pdfbookmark[2]{#1}{#1}
	\addtocontents{toc}{\vspace{-0pt}}
	\addcontentsline{toc}{chapter}{\hspace{14.5mm}\textbf{\appendixname~
	\thechapter~- #1}}
	\chapter*{\appendixname\space\space\thechapter~- \uppercase{#1}}%
	{}
}
\newcommand{\refap}[1]{\hyperref[#1]{Apêndice~\ref{#1}}} 	% Referência apÊndices
% uso do tikz e pgfplots
% ----------------------------------------------------------
%\usetikzlibrary{external}
\usetikzlibrary{arrows,calc,patterns,angles,quotes}
\usepackage{pgfplots}
\pgfplotsset{compat=1.15}

% Define o comando para citação de fontes em elementos gráficos (figuras, imagens,...).
% ----------------------------------------------------------
%  AUTOR(ano)
%
% parâmetro é a bibkey da fonte

\newcommand{\citefg}[1]{~\citeauthor{#1}(\citeyear{#1})}
