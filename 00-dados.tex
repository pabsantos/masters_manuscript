% !TeX root = ./main.tex

%%%%%%%%%%%%%%%%%%%%%%%%%%%%%%%%%%%%%%%%%%%%%%%%%%%%%%%
% Arquivo para entrada de dados para a parte pré textual
%%%%%%%%%%%%%%%%%%%%%%%%%%%%%%%%%%%%%%%%%%%%%%%%%%%%%%%
%
% Basta digitar as informações indicidas, no formato
% apresentado.
%
%%%%%%%
% Os dados solicitados são, na ordem:
%
% tipo do trabalho
% componentes do trabalho
% título do trabalho
% nome do autor
% local
% data (ano com 4 dígitos)
% orientador(a)
% coorientador(a)(as)(es)
% arquivo com dados bibliográficos
% instituição
% setor
% programa de pós gradução
% curso
% preambulo
% data defesa
% CDU
% errata
% assinaturas - termo de aprovação
% resumos & palavras chave
% agradecimentos
% dedicatoria
% epígrafe


% Informações de dados para CAPA e FOLHA DE ROSTO
%-----------------------------------------------------------------------------
\tipotrabalho{Dissertação}
%    {Relatório Técnico}
%    {Dissertação}
%    {Tese}
%    {Monografia}

% Marcar Sim para as partes que irão compor o documento pdf
%-----------------------------------------------------------------------------
\providecommand{\terCapa}{Sim}
\providecommand{\terFolhaRosto}{Sim}
\providecommand{\terTermoAprovacao}{Nao}
\providecommand{\terDedicatoria}{Nao}
\providecommand{\terFichaCatalografica}{Nao}
\providecommand{\terEpigrafe}{Nao}
\providecommand{\terAgradecimentos}{Nao}
\providecommand{\terErrata}{Nao}
\providecommand{\terListaFiguras}{Sim}
\providecommand{\terListaQuadros}{Nao}
\providecommand{\terListaTabelas}{Sim}
\providecommand{\terSiglasAbrev}{Nao}
\providecommand{\terResumos}{Sim}
\providecommand{\terSumario}{Sim}
\providecommand{\terAnexo}{Sim}
\providecommand{\terApendice}{Sim}
\providecommand{\terIndiceR}{Nao}
%-----------------------------------------------------------------------------

\titulo{Title: Subtitle}
\autor{Pedro Augusto Borges dos Santos}
\local{Curitiba}
\data{2021} %Apenas ano 4 dígitos

% Orientador ou Orientadora
\orientador{Prof. Dr. Jorge Tiago Bastos}
%Prof Emílio Eiji Kavamura, MSc}
\orientadora{}
% Pode haver apenas uma orientadora ou um orientador
% Se houver os dois prevalece o feminino.

% Em termos de coorientação, podem haver até quatro neste modelo
% Sendo 2 mulhere e 2 homens.
% Coorientador ou Coorientadora
\coorientador{Prof. Dr. Oscar Oviedo-Trespalacios}%Prof Morgan Freeman, DSc}
\coorientadora{}

% Segundo Coorientador ou Segunda Coorientadora
\scoorientador{}
%Prof Jack Nicholson, DEng}
\scoorientadora{}
%Prof\textordfeminine~Ingrid Bergman, DEng}
% ----------------------------------------------------------
\addbibresource{referencias.bib}

% ----------------------------------------------------------
\instituicao{%
Universidade Federal do Paraná}

\def \ImprimirSetor{}%
%Setor de Tecnologia}

\def \ImprimirProgramaPos{}%Programa de Pós Graduação em Engenharia de Construção Civil}

\def \ImprimirCurso{}%
%Curso de Engenharia Civil}

\preambulo{

Document presented as the qualifying paper for the obtaining of a degree in Master of Sciences by the Programa de Pós-Graduação em Planejamento Urbano of Setor de Tecnologia of Universidade Federal do Paraná }

%do grau de Bacharel em Expressão Gráfica no curso de Expressão Gráfica, Setor de Exatas da Universidade Federal do Paraná}

%-----------------------------------------------------------------------------

\newcommand{\imprimirCurso}{}
%Programa de P\'os Gradua\c{c}\~ao em Engenharia da Constru\c{c}\~ao Civil}

\newcommand{\imprimirDataDefesa}{}

\newcommand{\imprimircdu}{}

% ----------------------------------------------------------
\newcommand{\imprimirerrata}{}

% Comandos de dados - Data da apresentação
\providecommand{\imprimirdataapresentacaoRotulo}{}
\providecommand{\imprimirdataapresentacao}{}
\newcommand{\dataapresentacao}[2][\dataapresentacaoname]{\renewcommand{\dataapresentacao}{#2}}

% Comandos de dados - Nome do Curso
\providecommand{\imprimirnomedocursoRotulo}{}
\providecommand{\imprimirnomedocurso}{}
\newcommand{\nomedocurso}[2][\nomedocursoname]
  {\renewcommand{\imprimirnomedocursoRotulo}{#1}
\renewcommand{\imprimirnomedocurso}{#2}}


% ----------------------------------------------------------
\newcommand{\AssinaAprovacao}{

\assinatura{%\textbf
   {Professora} \\ UFPR}
   \assinatura{%\textbf
   {Professora} \\ ENSEADE}
   \assinatura{%\textbf
   {Professora} \\ TIT}
   %\assinatura{%\textbf{Professor} \\ Convidado 4}

   \begin{center}
    \vspace*{0.5cm}
    %{\large\imprimirlocal}
    %\par
    %{\large\imprimirdata}
    \imprimirlocal, \imprimirDataDefesa.
    \vspace*{1cm}
  \end{center}
  }

% ----------------------------------------------------------
%\newcommand{\Errata}{%\color{blue}
%Elemento opcional da \textcite[4.2.1.2]{NBR14724:2011}. Exemplo:
%}

% ----------------------------------------------------------
\newcommand{\EpigrafeTexto}{}

% ----------------------------------------------------------
\newcommand{\ResumoTexto}{%\color{blue}
Resumo
}

\newcommand{\PalavraschaveTexto}{%\color{blue}
Palavras-chave}

% ----------------------------------------------------------
\newcommand{\AbstractTexto}{%\color{blue}
Abstract
}
% ---
\newcommand{\KeywordsTexto}{%\color{blue}
Key-words
}

% ----------------------------------------------------------
\newcommand{\Resume}
{%\color{blue}
%Il s'agit d'un résumé en français.
}
% ---
\newcommand{\Motscles}
{%\color{blue}
 %latex. abntex. publication de textes.
}

% ----------------------------------------------------------
\newcommand{\Resumen}
{%\color{blue}
%Este es el resumen en español.
}
% ---
\newcommand{\Palabrasclave}
{%\color{blue}
%latex. abntex. publicación de textos.
}

% ----------------------------------------------------------
\newcommand{\AgradecimentosTexto}{}

% ----------------------------------------------------------
\newcommand{\DedicatoriaTexto}{%\color{blue}
}
