% !TeX root = ./main.tex

%%%%%%%%%%%%%%%%%%%%%%%%%%%%%%%%%%%%%%%%%%%%%%%%%%%%%%%
% Arquivo para entrada de dados para a parte pré textual
%%%%%%%%%%%%%%%%%%%%%%%%%%%%%%%%%%%%%%%%%%%%%%%%%%%%%%%
%
% Basta digitar as informações indicidas, no formato
% apresentado.
%
%%%%%%%
% Os dados solicitados são, na ordem:
%
% tipo do trabalho
% componentes do trabalho
% título do trabalho
% nome do autor
% local
% data (ano com 4 dígitos)
% orientador(a)
% coorientador(a)(as)(es)
% arquivo com dados bibliográficos
% instituição
% setor
% programa de pós gradução
% curso
% preambulo
% data defesa
% CDU
% errata
% assinaturas - termo de aprovação
% resumos & palavras chave
% agradecimentos
% dedicatoria
% epígrafe


% Informações de dados para CAPA e FOLHA DE ROSTO
%-----------------------------------------------------------------------------
\tipotrabalho{Dissertação}
%    {Relatório Técnico}
%    {Dissertação}
%    {Tese}
%    {Monografia}

% Marcar Sim para as partes que irão compor o documento pdf
%-----------------------------------------------------------------------------
\providecommand{\terCapa}{Sim}
\providecommand{\terFolhaRosto}{Sim}
\providecommand{\terTermoAprovacao}{Sim}
\providecommand{\terDedicatoria}{Nao}
\providecommand{\terFichaCatalografica}{Nao}
\providecommand{\terEpigrafe}{Nao}
\providecommand{\terAgradecimentos}{Sim}
\providecommand{\terErrata}{Nao}
\providecommand{\terListaFiguras}{Sim}
\providecommand{\terListaQuadros}{Nao}
\providecommand{\terListaTabelas}{Sim}
\providecommand{\terSiglasAbrev}{Nao}
\providecommand{\terResumos}{Sim}
\providecommand{\terSumario}{Sim}
\providecommand{\terAnexo}{Nao}
\providecommand{\terApendice}{Sim}
\providecommand{\terIndiceR}{Nao}
%-----------------------------------------------------------------------------

\titulo{THE IMPACT OF BUILT ENVIRONMENT ON SPEEDING behavior IN CURITIBA - BRAZIL}
\autor{Pedro Augusto Borges dos Santos}
\local{Curitiba}
\data{2022} %Apenas ano 4 dígitos

% Orientador ou Orientadora
\orientador{Prof. Dr. Jorge Tiago Bastos}
%Prof Emílio Eiji Kavamura, MSc}
\orientadora{}
% Pode haver apenas uma orientadora ou um orientador
% Se houver os dois prevalece o feminino.

% Em termos de coorientação, podem haver até quatro neste modelo
% Sendo 2 mulhere e 2 homens.
% Coorientador ou Coorientadora
\coorientador{Dr. Oscar Oviedo-Trespalacios}%Prof Morgan Freeman, DSc}
\coorientadora{}

% Segundo Coorientador ou Segunda Coorientadora
\scoorientador{}
%Prof Jack Nicholson, DEng}
\scoorientadora{}
%Prof\textordfeminine~Ingrid Bergman, DEng}
% ----------------------------------------------------------
\addbibresource{referencias.bib}

% ----------------------------------------------------------
\instituicao{%
Universidade Federal do Paraná}

\def \ImprimirSetor{}%
%Setor de Tecnologia}

\def \ImprimirProgramaPos{}%Programa de Pós Graduação em Engenharia de Construção Civil}

\def \ImprimirCurso{}%
%Curso de Engenharia Civil}

\preambulo{

Manuscript presented as requirement for the obtaining of a degree in Master of Sciences by the Postgraduate Program in Urban Planning of the Technology Sector of Federal University of Paraná}

%do grau de Bacharel em Expressão Gráfica no curso de Expressão Gráfica, Setor de Exatas da Universidade Federal do Paraná}

%-----------------------------------------------------------------------------

\newcommand{\imprimirCurso}{}
%Programa de P\'os Gradua\c{c}\~ao em Engenharia da Constru\c{c}\~ao Civil}

\newcommand{\imprimirDataDefesa}{}

\newcommand{\imprimircdu}{}

% ----------------------------------------------------------
\newcommand{\imprimirerrata}{}

% Comandos de dados - Data da apresentação
\providecommand{\imprimirdataapresentacaoRotulo}{}
\providecommand{\imprimirdataapresentacao}{}
\newcommand{\dataapresentacao}[2][\dataapresentacaoname]{\renewcommand{\dataapresentacao}{#2}}

% Comandos de dados - Nome do Curso
\providecommand{\imprimirnomedocursoRotulo}{}
\providecommand{\imprimirnomedocurso}{}
\newcommand{\nomedocurso}[2][\nomedocursoname]
  {\renewcommand{\imprimirnomedocursoRotulo}{#1}
\renewcommand{\imprimirnomedocurso}{#2}}


% ----------------------------------------------------------
\newcommand{\AssinaAprovacao}{

\assinatura{%\textbf
   {Professora} \\ UFPR}
   \assinatura{%\textbf
   {Professora} \\ ENSEADE}
   \assinatura{%\textbf
   {Professora} \\ TIT}
   %\assinatura{%\textbf{Professor} \\ Convidado 4}

   \begin{center}
    \vspace*{0.5cm}
    %{\large\imprimirlocal}
    %\par
    %{\large\imprimirdata}
    \imprimirlocal, \imprimirDataDefesa.
    \vspace*{1cm}
  \end{center}
  }

% ----------------------------------------------------------
%\newcommand{\Errata}{%\color{blue}
%Elemento opcional da \textcite[4.2.1.2]{NBR14724:2011}. Exemplo:
%}

% ----------------------------------------------------------
\newcommand{\EpigrafeTexto}{}

% ----------------------------------------------------------
\newcommand{\ResumoTexto}{%\color{blue}

O comportamento de motoristas relacionado ao excesso de velocidade é um dos principais fatores de risco de sinistros de trânsito, influenciando na gravidade e no risco desses sinistros. A maioria dos sinistros de trânsito e conflitos ocorrem em áreas urbanas. As características do desenvolvimento espacial de uma cidade, incluindo o ambiente construído, podem influenciar o desempenho da segurança viária nessas áreas. O objetivo principal desta pesquisa é investigar a influência do ambiente construído na segurança viária, utilizando a cidade de Curitiba, Brasil, como cenário do estudo. O ambiente construído consiste em características físicas dentro da cidade, incluindo padrões de desenvolvimento e características das vias, e pode ser dividido em seis categorias: densidade, diversidade, design, acessibilidade ao destino, distância ao trânsito e dados demográficos. Taxa de excesso de velocidade foi usada como um indicador de desempenho de segurança viária. O modelo estatístico de Regressão Geograficamente Ponderada (RGP) foi utilizado para explorar a correlação entre as variáveis do ambiente construído (AC) e a ocorrência de excesso de velocidade, devido à capacidade do modelo de analisar cenários que são espacialmente não estacionários. Os dados de velocidade foram coletados como parte de um Estudo Naturalístico de Condução (ENC). O ENC consiste em monitorar o comportamento dos motoristas em seus próprios veículos, tanto em condições normais como em condições críticas de segurança, com a utilização de sensores GPS e câmeras. A amostra foi composta de 32 condutores, 1.002 viagens, 381,45 horas de condução e 9,443.83 km de distância percorrida em Curitiba e sua região metropolitana, entre 2019 e 2021. O modelo RGP foi aplicado usando as zonas de tráfego (ZTs) de Curitiba como nível zonal. Vários tipos de kernel e tamanhos de largura de banda foram testados. Os diagnósticos do modelo mostraram que alguns modelos RGP tiveram um desempenho melhor do que a regressão global, mas nenhum dos modelos pôde prever a heterogeneidade espacial associada à ocorrência do excesso de velocidade com um desempenho desejável. Apenas a variável “proporção de vias arteriais”, incluída na categoria de design, apresentou correlação com o excesso de velocidade com significância estatística de 95\% e com 100\% das ZTs apresentando uma correlação inversa em relação ao excesso de velocidade. Em conclusão, as características das vias arteriais de Curitiba em relação à infraestrutura e controle de tráfego podem estar influenciando na redução do excesso de velocidade. A aplicação de novos modelos estatísticos deve considerar uma análise dividida em diferentes horários e em regiões menores dentro do território de Curitiba. Outros níveis zonais também devem ser testados para verificar seu desempenho. 
}

\newcommand{\PalavraschaveTexto}{%\color{blue}
Ambiente construído. Excesso de velocidade. Segurança viária.}

% ----------------------------------------------------------
\newcommand{\AbstractTexto}{%\color{blue}

Speeding behavior performed by motorized vehicle drivers is one of the main risk factors of road crashes, influencing the severity and the risk of road crashes. Most traffic crashes and conflicts occur in urban areas. City development characteristics, including the built environment, can influence road safety performance in these areas. The main objective of this research is to investigate the influence of the built environment on road safety, using the city of Curitiba, Brazil, as the scenario of the study. The built environment consists of physical features inside the city, including development pattern and roadway design, and can be split into six categories: density, diversity, design, destination accessibility, distance to transit, and demographics. The speeding rate was used as a road safety performance indicator. The Geographically Weighted Regression (GWR) statistical model was used to explore the correlation between built environment (BE) variables and the occurrence of speeding, due to the model's ability to analyze scenarios that are spatially nonstationary. Speeding data was collected as part of a Naturalistic Driving Study (NDS). NDS consists of monitoring drivers' behavior in their own cars, both in normal and safety-critical conditions, with the use of onboard GPS sensors and cameras. In this work, the database contained 32 drivers, 1,002 trips, 381.45 hours of driving, and 9,443.83 km of traveled distance in Curitiba and its metropolitan area, between 2019 and 2021. The GWR model was applied using Curitiba's traffic analysis zones (TAZs) as zonal level. Multiple kernel types and bandwidth sizes were tested. The model diagnostics showed that some GWR performed better than the global regression, but none of the models had a desirable performance regarding the spatial heterogeneity associated with the occurrence of speeding. Only the variable ``proportion of arterial roads'', included in the design category, showed a correlation to speeding at a statistical significance of 95\%, with 100\% of TAZs presenting an inverted correlation to speeding. In conclusion, the characteristics of arterial roads in Curitiba regarding infrastructure and traffic control might be influencing the reduction of speeding behavior. Future statistical models should split the analysis into different time frames and into smaller regions inside Curitiba's territory. Other zonal systems should also be tested to verify their overall performance. 

}
% ---
\newcommand{\KeywordsTexto}{%\color{blue}
Built environment. Speeding behavior. Road safety.
}

% ----------------------------------------------------------
\newcommand{\Resume}
{%\color{blue}
%Il s'agit d'un résumé en français.
}
% ---
\newcommand{\Motscles}
{%\color{blue}
 %latex. abntex. publication de textes.
}

% ----------------------------------------------------------
\newcommand{\Resumen}
{%\color{blue}
%Este es el resumen en español.
}
% ---
\newcommand{\Palabrasclave}
{%\color{blue}
%latex. abntex. publicación de textos.
}

% ----------------------------------------------------------
\newcommand{\AgradecimentosTexto}{
The author would like to thank the funding by the National Council for Scientific and Technological Development (CNPq, Brazil)/MCTIC/CNPq N. 28/2018, Universal/Faixa A, by the Coordination for the Improvement of Higher Education Personnel (CAPES)/Finance Code 001 and by the National Observatory for Road Safety (ONSV) under the Technical Cooperation Agreement with the Federal University of Paraná.
}

% ----------------------------------------------------------
\newcommand{\DedicatoriaTexto}{%\color{blue}
}
