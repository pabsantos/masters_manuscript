% !TeX root = ./main.tex

%%%%%%%%%%%%%%%%%%%%%%%%%%%%%%%%%%%%%%%%%%%%%%%%%%%%%%%
% Arquivo para entrada de dados para a parte pré textual
%%%%%%%%%%%%%%%%%%%%%%%%%%%%%%%%%%%%%%%%%%%%%%%%%%%%%%%
%
% Basta digitar as informações indicidas, no formato
% apresentado.
%
%%%%%%%
% Os dados solicitados são, na ordem:
%
% tipo do trabalho
% componentes do trabalho
% título do trabalho
% nome do autor
% local
% data (ano com 4 dígitos)
% orientador(a)
% coorientador(a)(as)(es)
% arquivo com dados bibliográficos
% instituição
% setor
% programa de pós gradução
% curso
% preambulo
% data defesa
% CDU
% errata
% assinaturas - termo de aprovação
% resumos & palavras chave
% agradecimentos
% dedicatoria
% epígrafe


% Informações de dados para CAPA e FOLHA DE ROSTO
%-----------------------------------------------------------------------------
\tipotrabalho{Dissertação}
%    {Relatório Técnico}
%    {Dissertação}
%    {Tese}
%    {Monografia}

% Marcar Sim para as partes que irão compor o documento pdf
%-----------------------------------------------------------------------------
\providecommand{\terCapa}{Sim}
\providecommand{\terFolhaRosto}{Sim}
\providecommand{\terTermoAprovacao}{Nao}
\providecommand{\terDedicatoria}{Nao}
\providecommand{\terFichaCatalografica}{Nao}
\providecommand{\terEpigrafe}{Nao}
\providecommand{\terAgradecimentos}{Nao}
\providecommand{\terErrata}{Nao}
\providecommand{\terListaFiguras}{Sim}
\providecommand{\terListaQuadros}{Nao}
\providecommand{\terListaTabelas}{Sim}
\providecommand{\terSiglasAbrev}{Nao}
\providecommand{\terResumos}{Sim}
\providecommand{\terSumario}{Sim}
\providecommand{\terAnexo}{Sim}
\providecommand{\terApendice}{Sim}
\providecommand{\terIndiceR}{Nao}
%-----------------------------------------------------------------------------

\titulo{THE IMPACT OF BUILT ENVIRONMENT ON ROAD SAFETY IN CITIES}
\autor{Pedro Augusto Borges dos Santos}
\local{Curitiba}
\data{2021} %Apenas ano 4 dígitos

% Orientador ou Orientadora
\orientador{Prof. Dr. Jorge Tiago Bastos}
%Prof Emílio Eiji Kavamura, MSc}
\orientadora{}
% Pode haver apenas uma orientadora ou um orientador
% Se houver os dois prevalece o feminino.

% Em termos de coorientação, podem haver até quatro neste modelo
% Sendo 2 mulhere e 2 homens.
% Coorientador ou Coorientadora
\coorientador{Prof. Dr. Oscar Oviedo-Trespalacios}%Prof Morgan Freeman, DSc}
\coorientadora{}

% Segundo Coorientador ou Segunda Coorientadora
\scoorientador{}
%Prof Jack Nicholson, DEng}
\scoorientadora{}
%Prof\textordfeminine~Ingrid Bergman, DEng}
% ----------------------------------------------------------
\addbibresource{referencias.bib}

% ----------------------------------------------------------
\instituicao{%
Universidade Federal do Paraná}

\def \ImprimirSetor{}%
%Setor de Tecnologia}

\def \ImprimirProgramaPos{}%Programa de Pós Graduação em Engenharia de Construção Civil}

\def \ImprimirCurso{}%
%Curso de Engenharia Civil}

\preambulo{

Document presented as the qualifying paper for the obtaining of a degree in Master of Sciences by the Programa de Pós-Graduação em Planejamento Urbano of Setor de Tecnologia of Universidade Federal do Paraná }

%do grau de Bacharel em Expressão Gráfica no curso de Expressão Gráfica, Setor de Exatas da Universidade Federal do Paraná}

%-----------------------------------------------------------------------------

\newcommand{\imprimirCurso}{}
%Programa de P\'os Gradua\c{c}\~ao em Engenharia da Constru\c{c}\~ao Civil}

\newcommand{\imprimirDataDefesa}{}

\newcommand{\imprimircdu}{}

% ----------------------------------------------------------
\newcommand{\imprimirerrata}{}

% Comandos de dados - Data da apresentação
\providecommand{\imprimirdataapresentacaoRotulo}{}
\providecommand{\imprimirdataapresentacao}{}
\newcommand{\dataapresentacao}[2][\dataapresentacaoname]{\renewcommand{\dataapresentacao}{#2}}

% Comandos de dados - Nome do Curso
\providecommand{\imprimirnomedocursoRotulo}{}
\providecommand{\imprimirnomedocurso}{}
\newcommand{\nomedocurso}[2][\nomedocursoname]
  {\renewcommand{\imprimirnomedocursoRotulo}{#1}
\renewcommand{\imprimirnomedocurso}{#2}}


% ----------------------------------------------------------
\newcommand{\AssinaAprovacao}{

\assinatura{%\textbf
   {Professora} \\ UFPR}
   \assinatura{%\textbf
   {Professora} \\ ENSEADE}
   \assinatura{%\textbf
   {Professora} \\ TIT}
   %\assinatura{%\textbf{Professor} \\ Convidado 4}

   \begin{center}
    \vspace*{0.5cm}
    %{\large\imprimirlocal}
    %\par
    %{\large\imprimirdata}
    \imprimirlocal, \imprimirDataDefesa.
    \vspace*{1cm}
  \end{center}
  }

% ----------------------------------------------------------
%\newcommand{\Errata}{%\color{blue}
%Elemento opcional da \textcite[4.2.1.2]{NBR14724:2011}. Exemplo:
%}

% ----------------------------------------------------------
\newcommand{\EpigrafeTexto}{}

% ----------------------------------------------------------
\newcommand{\ResumoTexto}{%\color{blue}

Os sinistros de trânsito são causa de mais de 1,35 milhão de mortes e 50 milhões de feridos por ano, sendo a oitava causa de morte no mundo. As mortes no trânsito envolvendo pedestres, ciclistas e motociclistas representam mais da metade do valor total das mortes no trânsito global. O excesso de velocidade praticado por veículos motorizados é um dos principais fatores de risco dos sinistros de trânsito, influenciando na gravidade e na chance de ocorrência dos sinistros. A maioria dos sinistros de trânsito e conflitos acontecem nas cidades. As características do desenvolvimento espacial de uma cidade, incluindo o ambiente construído, podem influenciar o desempenho da segurança viária em áreas urbanas. O objetivo principal desta pesquisa é investigar a influência do ambiente construído na segurança viária, utilizando a cidade de Curitiba, Brasil, como cenário do estudo. O ambiente construído consiste em características físicas dentro da cidade, incluindo padrões de desenvolvimento e características das vias, e pode ser dividido em seis categorias: densidade, diversidade, design, acessibilidade ao destino, distância ao trânsito e dados demográficos. O excesso de velocidade foi usado como um indicador de desempenho de segurança viária. O modelo estatístico de Regressão Geograficamente Ponderada (RGP) foi utilizado para explorar a correlação entre as variáveis do ambiente construído (AC) e a ocorrência de excesso de velocidade. Trabalhos anteriores utilizaram o RGP para explorar a correlação entre o AC e os sinistros de trânsito, exibindo sua capacidade de analisar cenários que são espacialmente não estacionários. Os dados de velocidade foram coletados usando o método do Estudo Naturalístico de Condução (ENC). O ENC consiste em monitorar o comportamento dos condutores em seus próprios carros, tanto em condições normais quanto em condições críticas de segurança, com o uso de sensores GPS e câmeras equipadas nos carros dos voluntários. Neste trabalho, foram utilizados dados de 16 motoristas, contendo 491 viagens, 238,85 horas de condução e 5.362,75 km de distância percorrida em Curitiba e sua região metropolitana, no período entre 2019 e 2021. O modelo RGP foi aplicado usando as zonas de tráfico (ZTs) de Curitiba como unidade de análise. Vários tipos de kernel e tamanhos de largura de banda foram testados. Os diagnósticos do modelo mostraram que o RGP teve um desempenho melhor do que a regressão global, mas nenhum dos modelos pôde predizer validamente a heterogeneidade espacial presente na ocorrência do excesso de velocidade. Apenas a variável ``densidade de radares'', incluído na categoria design, apresentou correlação com o excesso de velocidade, com significância estatística de 95\% e com 97,2\% das ZTs apresentando coeficiente negativo, portanto, correlacionado invertidamente com o excesso de velocidade. A amostra atual é relativamente pequena para conduzir uma análise mais significativa. É importante testar novos modelos com uma amostra maior antes de chegar a mais conclusões.
}

\newcommand{\PalavraschaveTexto}{%\color{blue}
Ambiente construído. Excesso de velocidade. Segurança viária.}

% ----------------------------------------------------------
\newcommand{\AbstractTexto}{%\color{blue}

Road traffic crashes causes more than 1.35 million deaths and 50 million in injuries per year, being the eighth leading cause of death in the world. Road traffic deaths involving pedestrians, cyclists and motorcyclists represents more than half of the total value of global road traffic deaths. The excess of speed performed by motorized vehicles is one of the main risk factors of road crashes, influencing the severity and chance of occurrence of road crashes. Most of traffic crashes and conflicts happens in cities. City development characteristics, including the built environment, can influence the road safety performance in urban areas. The main objective of this research is to investigate the influence of the built environment in road safety, using the city of Curitiba, Brazil, as scenario of the study. The built environment consists of physical features inside the city, including development patterns and roadway designs, and can be split into six categories: density, diversity, design, destination accessibility, distance to transit and demographics. Speeding was used as a road safety performance indicator. The Geographically Weighted Regression (GWR) statistical model was used to explore the correlation between built environment (BE) variables and the occurrence of speeding. Previous works used GWR to explore the correlation between BE and road crashes, displaying it's ability to analyze scenarios that are spatially nonstationary. The speeding data was collected using the Naturalistic Driving Study (NDS) method. NDS consists in monitoring drivers behavior in their own cars, both in normal and safety critical conditions, with the use of GPS sensors and cameras equipped in the volunteer's cars. In this work, data from 16 drivers was used, containing 491 trips, 238.85 hours of driving and 5,362.75 km of distance travelled in Curitiba and it's metropolitan area, ranging between 2019 and 2021. The GWR model was applied using Curitiba's traffic analysis zones as zonal level. Multiple kernel types and bandwidths sizes were tested. The model diagnostics showed that GWR performed better than the global regression, but none of the models could validly predict the spatial heterogeneity present in the occurrence of speeding. Only the variable ``density of speed cameras'', included in the design category, showed a correlation to speeding, at a statistical significance of 95\%, with 97.2\% of TAZs presenting a negative coefficient, therefore, a inverted correlation to speeding. The current sample is relative small to conduct a more significant analysis. It is important to test new models with a bigger sample before reaching any further conclusions. 

}
% ---
\newcommand{\KeywordsTexto}{%\color{blue}
Built environment. Speeding. Road safety.
}

% ----------------------------------------------------------
\newcommand{\Resume}
{%\color{blue}
%Il s'agit d'un résumé en français.
}
% ---
\newcommand{\Motscles}
{%\color{blue}
 %latex. abntex. publication de textes.
}

% ----------------------------------------------------------
\newcommand{\Resumen}
{%\color{blue}
%Este es el resumen en español.
}
% ---
\newcommand{\Palabrasclave}
{%\color{blue}
%latex. abntex. publicación de textos.
}

% ----------------------------------------------------------
\newcommand{\AgradecimentosTexto}{}

% ----------------------------------------------------------
\newcommand{\DedicatoriaTexto}{%\color{blue}
}
